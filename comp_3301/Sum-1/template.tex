\documentclass{article}[12pt]
\pagestyle{empty}
\linespread{1.2}

\begin{document}

\title{\vspace{-4cm}
The Moral Challenges of Driverless Cars}
\author{Christopher Blackman 100992827}
\date{COMP 3301, Sept 11, 2017}

\maketitle

\pagenumbering{gobble}


Driverless cars are no longer fiction and many questions must be answered about them.
When in control of a vehicle, there are many moral decisions associated with how one operates the machine.
Based on these day-to-day moral decisions, computers have sensors that react based on pre-programmed logic to these decisions.
Yet, theses preprogrammed logic machines have far greater reaction times, perception abilities and do not suffer from symptoms of human fatigue or conditions.
Roughly ninety percent of car crashes are cause by human error and by introducing driverless cars to solve this problem without generating more entropy, would be the scenario of least consequences.
However, should an unavoidable incident occur, vision technology is not yet at the stage where it can identify objects and react based on morals.
Detection technology used in modern cars only see numerical values based on brightness of pixels; it must then infer what theses pixels are, which is easy for the average human.
In time detection technology will be able to recognize more complex objects, but as of the moment only simple objects are recognizable with a high degree of accuracy.
Once detection technology becomes adequate; blaming inadequacy of technology is no longer viable excuse to restrict driverless cars, thus more responsible ethical decisions must be contrived.
This is a difficult answer as drivers are often faced with less than ideal options in unavoidable situations.
This can be seen in common ethics problems like the 'trolley problem', where one must choose to save the one, the driver, or the many.
These choices will likely not end up with the end user, due to natural instincts. 
A solution to this is to require mandatory potential crash scenarios with a report on the reactions of the vehicle; this would be reviewed by some kind of review board.
Thus, instead of quick laws in response to driverless cars, other communities can work with the review board during development of driverless cars.
Overtime consensus over the technology may emerge and will be reflected in rules and legislation.
Solving theses ethical issues en masse will not solve all ethical problem on this subject, but even in modern cars there are still unresolved ethical issues.
Furthermore, there are many automated technologies in modern cars that are ethically unregulated.
The larger companies pressing for driverless cars are unlikely to wait for all ethical issue solutions; they will most likely follow the principles of safe driving.
Another solution is to give control back to the driver, allowing moral decision making, in emergency situations, although be doing so the driver will be at a complete disadvantage due to lack of situational awareness.
Thus the reason for the push to fully automated cars, however there is still ways to go since driving performance is still poor in less than optimal conditions.
As time increases and bugs are worked out, car manufacturers will have more time to work through these ethical issues and as technology increases as with awareness, moral decisions will be made less frequent.


\end{document}

